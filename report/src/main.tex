
% Preamble
\documentclass[11pt]{report}

% Packages
\usepackage{amsmath}
\usepackage{graphicx}
\usepackage{color}
\usepackage[table,xcdraw]{xcolor}
\usepackage{multirow}
\usepackage{float}


\usepackage[margin=2.5cm]{geometry}



% set image file paths
%\graphicspath{ {../../fuzzy/output/mamdani\_bell\_v9/io\_graphs/} }


\title{Fuzzy Systems and Neural Networks}
\author{Jéssica Consciência e Tiago Leite}


\begin{document}
\maketitle
\newpage

\part{Fuzzy System}

%% Texto ainda por melhorar!!
Firstly we started by deciding between which type of fuzzy system
we should implement: Mamdani, Takagi-Sugeno or Tsukamoto. From the
project statement we observe that the output \textit{CLPVariation}
is not any clear function of the input, rulling out Takagi-Sugeno,
also meaning that our output is a \textbf{Fuzzy Set}. If we wish for
our output to be monotonic then the choice would be Tsukomoto, since
we did not want this restriction and decided for starting with a simple
approach then later on adding difficulty when needed.
(Early on we decided to try to make data-driven decisions with an iterative
improving process)

\section{First Iterations}

In the initial iteration, we selected the variables \textit{ProcessorLoad}, \textit{MemoryUsage}, and \textit{Latency} based on common sense.
These variables were chosen as inputs, while \textit{CLP} was designated as the output.
We opted for triangular membership functions, defining four levels for each input variable: (low, medium, high, critical) for \textit{ProcessorLoad} and \textit{MemoryUsage}, and (poor, fair, good, great) for \textit{Latency}.

To start building the system, we decided to focus on just two variables: \textit{MemoryUsage} and \textit{ProcessorLoad}.
We then defined the range of the membership functions associated with each term of the two linguistic variables.
Considering that a device with more than 85\% processor load or memory usage is typically unable to perform its basic tasks,
it became clear that this threshold would correspond to a specific term, which we labeled as ``critical''. The ranges for the other membership function terms were distributed between 0 and 1 based on what we deemed appropriate.
We also decided to keep the terms associated with \textit{CLP} straightforward, using only three terms: ``decrease'', ``increase'', and ``maintain''.
The values for the membership functions of these terms were distributed between -1 and 1.

The figures below illustrate the membership function graphs for these variables.

\begin{figure}[htbp]
    \centering
    \begin{minipage}{0.32\textwidth}
        \centering
        \includegraphics[width=\textwidth]{../images/triangular_MemoryUsage}
        \caption{Memory Usage}
        \label{fig:processor_load}
    \end{minipage}
    \hfill
    \begin{minipage}{0.32\textwidth}
        \centering
        \includegraphics[width=\textwidth]{../images/triangular_ProcessorLoad}
        \caption{Processor Load}
        \label{fig:memory_usage}
    \end{minipage}
    \hfill
    \begin{minipage}{0.32\textwidth}
        \centering
        \includegraphics[width=\textwidth]{../images/triangular_CLP}
        \caption{CLP Variation}
        \label{fig:clp}
    \end{minipage}
\end{figure}


To design the system's rules, we created a truth table, which can be found below in Table~\ref{truthtable}.
The logic behind the table was as follows: when both \textit{MemoryUsage} and \textit{ProcessorLoad} were either "low" or "medium," the \textit{CLP} would increase.
When one of them reached "high," the \textit{CLP} remained unchanged (this decision was made to ensure that the node's processing capacity stayed above average).
Finally, if any of these variables entered a "critical" state, the \textit{CLP} had to decrease.


\begin{table}[H][htbp]
    \centering
    \begin{tabular}{|
    >{\columncolor[HTML]{9698ED}}c
    >{\columncolor[HTML]{CBCEFB}}c |cccc|}
    \hline
    \multicolumn{2}{|c|}{\cellcolor[HTML]{FFFC9E}}                                         & \multicolumn{4}{c|}{\cellcolor[HTML]{9698ED}ProcessorLoad}                                                                                                                                    \\ \cline{3-6}
    \multicolumn{2}{|c|}{\multirow{-2}{*}{\cellcolor[HTML]{FFFC9E}CPL}}                    & \multicolumn{1}{c|}{\cellcolor[HTML]{CBCEFB}low} & \multicolumn{1}{c|}{\cellcolor[HTML]{CBCEFB}medium} & \multicolumn{1}{c|}{\cellcolor[HTML]{CBCEFB}high} & \cellcolor[HTML]{CBCEFB}critical \\ \hline
    \multicolumn{1}{|c|}{\cellcolor[HTML]{9698ED}}                              & low      & \multicolumn{1}{c|}{increase}                    & \multicolumn{1}{c|}{increase}                       & \multicolumn{1}{c|}{mantain}                      & decrease                         \\ \cline{2-6}
    \multicolumn{1}{|c|}{\cellcolor[HTML]{9698ED}}                              & medium   & \multicolumn{1}{c|}{increase}                    & \multicolumn{1}{c|}{increase}                       & \multicolumn{1}{c|}{maintain}                     & decrease                         \\ \cline{2-6}
    \multicolumn{1}{|c|}{\cellcolor[HTML]{9698ED}}                              & high     & \multicolumn{1}{c|}{maintain}                    & \multicolumn{1}{c|}{maintain}                       & \multicolumn{1}{c|}{maintain}                     & Decrease                         \\ \cline{2-6}
    \multicolumn{1}{|c|}{\multirow{-4}{*}{\cellcolor[HTML]{9698ED}MemoryUsage}} & critical & \multicolumn{1}{c|}{Decrease}                    & \multicolumn{1}{c|}{Decrease}                       & \multicolumn{1}{c|}{Decrease}                     & Decrease                         \\ \hline
    \end{tabular}
    \caption{Truth table}
    \label{truthtable}

\end{table}



To visualize the system's output, we generated 50 data points for \textit{MemoryUsage} and \textit{ProcessorLoad} ranging between 0 and 1.
We then created an interactive 3D graph that showed the evolution of \textit{CLP} based on these two values.
This graph can be seen in Figure~\ref{fig:3d_triangular}.
Upon reviewing the graph, we noticed that the variables \textit{ProcessorLoad} and \textit{MemoryUsage} exhibited very similar behavior.
Intuitively, when designing the system, we had structured the membership functions for each term in the same way for both variables, and the truth table was also symmetric.
This indicates that the system should react in the same way to both variables and they could, in fact, be merged into a single variable without losing the system's effectiveness.
By combining these two variables, we simplify the model while still accurately representing the system's behavior, as both variables seem to influence the \textit{CLP} in a nearly identical manner.


\begin{figure}[h]
    \centering
\includegraphics[scale = 0.8]{../images/3d_triangular}
\caption{Fuzzy CLP Inference}
\label{fig:3d_triangular}
\end{figure}



Subsequently, we explored the effect of switching the membership functions to a Gaussian distribution.


During the experimentation phase with different membership functions,the need for visualization became apparent.
To facilitate this, we developed a helper script [fuzzy/visualization/fuzzy\_system\_to\_dataframe] that converts the
FuzzySystem Python object into a dynamic dataframe, enabling easy plotting and analysis of the membership functions.



\section{Generalized Bell}
We decided to experiment with a more generic Membership function, so we
extended simpful's Base Membership Function class and created Bell\_MF [in fuzzy/models/bell\_mf.py].
The first results are shown in the figure bellow.


\begin{figure}[htbp]
    \centering
    \begin{minipage}{0.32\textwidth}
        \centering
        \includegraphics[width=\textwidth]{../images/first_bell/MemoryUsage}
        \caption{Memory Usage}
        \label{fig:first_bell_processor_load}
    \end{minipage}
    \hfill
    \begin{minipage}{0.32\textwidth}
        \centering
        \includegraphics[width=\textwidth]{../images/first_bell/ProcessorLoad}
        \caption{Processor Load}
        \label{fig:first_bell_memory_usage}
    \end{minipage}
    \hfill
    \begin{minipage}{0.32\textwidth}
        \centering
        \includegraphics[width=\textwidth]{../images/first_bell/CLP}
        \caption{CLP Variation}
        \label{fig:first_bell_clp}
    \end{minipage}
\end{figure}


%\includegraphics[width=0.5]{../../fuzzy/output/mamdani_bell/io_graphs/MemoryUsage.png}
%\includegraphics[width=0.5]{../../fuzzy/output/mamdani_bell/io_graphs/ProcessorLoad.png}
%\includegraphics[width=0.5]{../../fuzzy/output/mamdani_bell/io_graphs/CLP.png}

%%
\section{Architecture}
This should contain choice of architecture and why.

\section{Membership Functions}
all the membership functions and linguistic terms

\begin{figure}

%\includegraphics{CLP.png}
%\begin{tabular}[ccc]
%    \subfloat[caption]{\includegraphics[width=.3\textwidth]{CLP.png}} & % noqa
%    \subfloat[caption]{\includegraphics[width=.3\textwidth]{CLP.png}} &
%    \subfloat[caption]{\includegraphics[width=.3\textwidth]{CLP.png}} \\
%\end{tabular}
\end{figure}

\section{Rules}
rules

\begin{table}[H]
\begin{tabular}{|
>{\columncolor[HTML]{9698ED}}c l|cccc|}
\hline
\multicolumn{2}{|c|}{\cellcolor[HTML]{FFCC67}{\color[HTML]{333333} }}                                & \multicolumn{4}{c|}{\cellcolor[HTML]{9698ED}Latency}                                                        \\ \cline{3-6}
\multicolumn{2}{|c|}{\multirow{-2}{*}{\cellcolor[HTML]{FFCC67}{\color[HTML]{333333} CLP Variation}}} & \multicolumn{1}{l|}{low} & \multicolumn{1}{l|}{moderate} & \multicolumn{1}{l|}{high} & \multicolumn{1}{l|}{very high} \\ \hline
\multicolumn{1}{|c|}{\cellcolor[HTML]{9698ED}}                                     & low             & \multicolumn{1}{c|}{IS}  & \multicolumn{1}{c|}{IS}       & \multicolumn{1}{c|}{I}    & I                              \\ \cline{2-6}
\multicolumn{1}{|c|}{\cellcolor[HTML]{9698ED}}                                     & moderate        & \multicolumn{1}{c|}{I}   & \multicolumn{1}{c|}{I}        & \multicolumn{1}{c|}{I}    & I                              \\ \cline{2-6}
\multicolumn{1}{|c|}{\cellcolor[HTML]{9698ED}}                                     & high            & \multicolumn{1}{c|}{M}   & \multicolumn{1}{c|}{M}        & \multicolumn{1}{c|}{D}    & D                              \\ \cline{2-6}
\multicolumn{1}{|c|}{\multirow{-4}{*}{\cellcolor[HTML]{9698ED}System Load}}        & critical        & \multicolumn{1}{c|}{DS}  & \multicolumn{1}{c|}{DS}       & \multicolumn{1}{c|}{DS}   & DS                             \\ \hline
\end{tabular}
\end{table}


\section{Results}


\part{Neural Networks}



\end{document}
